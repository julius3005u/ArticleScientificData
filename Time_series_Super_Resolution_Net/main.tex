%% 
%% Copyright 2019-2024 Elsevier Ltd
%% 
%% Version 2.4
%% 
%% This file is part of the 'CAS Bundle'.
%% --------------------------------------
%% 
%% It may be distributed under the conditions of the LaTeX Project Public
%% License, either version 1.2 of this license or (at your option) any
%% later version.  The latest version of this license is in
%%    http://www.latex-project.org/lppl.txt
%% and version 1.2 or later is part of all distributions of LaTeX
%% version 1999/12/01 or later.
%% 
%% The list of all files belonging to the 'CAS Bundle' is
%% given in the file `manifest.txt'.
%% 
%% Template article for cas-sc documentclass for 
%% single column output.

%\documentclass[a4paper,fleqn,longmktitle]{cas-sc}
\documentclass[a4paper,fleqn]{cas-sc}

\usepackage[numbers]{natbib}
% \usepackage[authoryear]{natbib}
% \usepackage[authoryear,longnamesfirst]{natbib}
% \usepackage{todonotes}
\usepackage[caption=false,font=footnotesize]{subfig}

%%%Author macros
\def\tsc#1{\csdef{#1}{\textsc{\lowercase{#1}}\xspace}}
\tsc{WGM}
\tsc{QE}
\tsc{EP}
\tsc{PMS}
\tsc{BEC}
\tsc{DE}
%%%

\begin{document}
\let\WriteBookmarks\relax
\def\floatpagepagefraction{1}
\def\textpagefraction{.001}

\title[mode = title]{Super-Resolution: An Exploratory Analysis Based on Synthetic Data}
\shorttitle{Super-Resolution: An Exploratory Analysis Based on Synthetic Data}

%\begin{frontmatter}
\begin{graphicalabstract}
    \includegraphics[width=\linewidth]{images/graphical_abstract.pdf}
\end{graphicalabstract}

\begin{highlights}
    \item Synthetic data can effectively support 1D signal super-resolution when combined with real data or used for pretraining.
    \item Pretraining on synthetic datasets enables representation transfer across domains, providing a viable solution when real high-resolution signals are scarce.
    \item The synthetic data does not resemble real data to effectively aid super-resolution tasks. Being highly variable and non-stationary appears sufficient.
\end{highlights}

%%%%%%%%%%%%%%%% Authors %%%%%%%%%%%%%%%% 
\author[1]{Julio Ibarra Fiallo}[
    % type=editor,
    role=Researcher,
    % auid=000,
    % bioid=1,
    % orcid=0000-0001-0000-0000
]
\cormark[1]
\ead{jibarra@usfq.edu.ec}
% \ead[URL]{www.campus.in}
\credit{Conceptualization of this study, Methodology, Software}

\author[2]{Juan A. Lara}[
   role=Researcher,
]
\ead{correo@pendiente}
\credit{Data curation, Writing - Original draft preparation}

\author[1]{José Ocampo}[
   role=Researcher,
]
\ead{jocampor@estud.usfq.edu.ec}

%%%%%%%%%%%%%%%% Affiliations %%%%%%%%%%%%%%%%
\affiliation[1]{
    organization={Universidad San Francisco de Quito},
    addressline={Diego de Robles y Vía Interoceánica}, 
    city={Quito},
    %citysep={}, % Uncomment if no comma needed between city and postcode
    postcode={170901}, 
    state={Pichincha},
    country={Ecuador}
}

\affiliation[2]{
    organization={Universidad de Córdoba},
    addressline={Campus de Rabanales. Edificio Albert Einstein. Ctra. Madrid, Km 396}, 
    postcode={14071}, 
    city={Córdoba},
    country={España}
}

%%%%%%%%%%%%%%%% Author notes %%%%%%%%%%%%%%%%
\cortext[cor1]{Corresponding author}
% \fntext[fn1]{This is the first author footnote, but is common to third
%   author as well.}
% \fntext[fn2]{Another author footnote, this is a very long footnote and
%   it should be a really long footnote. But this footnote is not yet
%   sufficiently long enough to make two lines of footnote text.}

%%%%%%%%%%%%%%%% Short authors abbreviations %%%%%%%%%%%%%%%%
\shortauthors{Ibarra et~al.}

%%%%%%%%%%%%%%%% Paper content %%%%%%%%%%%%%%%%
\begin{abstract}[]
    Super-resolution (SR) aims to reconstruct high-resolution (HR) signals from low-resolution (LR) observations. Deep learning methods have advanced this task, yet they rely on abundant HR data that can be scarce, costly or hard to obtain. This study investigates the use of synthetic data to train SR models for one-dimensional (1D) signals. Using EEG recordings and synthetically generated signals, we evaluate four training strategies: training on the real dataset only (Real), training exclusively with synthetic data (Synthetic), training on both synthetic and real data jointly (Mixed), and synthetic pretraining followed by real fine-tuning (Tunned).

Synthetic-only models perform worst across datasets, while combining or pretraining with synthetic data improves results substantially. On EEG validation data, the Mixed model reduces mean absolute error (MAE) by 9.64\% relative to the Real baseline; on the out-of-domain VCTK speech dataset, the Tunned model achieves a 25.51\% reduction. These findings show that synthetic data effectively augment limited real datasets, enhancing generalization and robustness in SR tasks.
\end{abstract}


\begin{keywords}
    Super-resolution \sep Synthetic data \sep Deep learning \sep Data Augmentation
\end{keywords}


% \listoftodos
% \newpage

\maketitle

%----------------------------------------------------------------------------------------
%	ARTICLE CONTENTS
%----------------------------------------------------------------------------------------

\section{Introduction}
The resolution of one-dimensional (1D) signals plays a crucial role in a wide range of scientific and industrial domains, from biomedical monitoring to audio processing and network analysis. However, limitations in hardware, energy constraints, and transmission bottlenecks often force the collection of signals at suboptimal sampling rates. This undersampling can obscure short-lived but critical events, reducing the performance of downstream tasks such as classification or anomaly detection. Super-resolution (SR) techniques aim to mitigate this by increasing the number of points in a signal without introducing distortion or losing original information.

\subsection{Related work}
Classical SR methods have traditionally relied on interpolation and signal processing techniques, but these often fail to reconstruct high-frequency components or preserve structural consistency in complex data. In contrast, deep learning approaches have recently emerged as state-of-the-art solutions by learning to generate plausible high-resolution reconstructions from coarse inputs.

One of the earliest deep learning approaches for 1D signal super-resolution was proposed by \cite{kuleshov2017audio}, who demonstrated that a convolutional neural network (CNN) could upsample speech and music audio by factors ranging from $2\times$ to $6\times$, outperforming traditional interpolation techniques. CNN-based architectures have since been adapted for other domains. For example, \cite{kaniraja2024deep} proposed ECG-SRCNN, a model capable of $4\times$ upsampling of electrocardiogram (ECG) signals. Their approach not only improved signal reconstruction but also enhanced arrhythmia classification accuracy to 99\%.

Generative Adversarial Networks (GANs) have also proven effective for SR by encouraging the generator to produce more realistic outputs through adversarial training. \cite{hu2020phase} developed a phase-aware GAN model for music super-resolution, reconstructing both magnitude and phase of the spectrogram and showing significant perceptual improvements. \cite{kim2018adversarial} explored adversarial training with unsupervised feature losses for general audio SR, achieving high fidelity in upsampled signals. \cite{luo2020eeg} proposed a Wasserstein GAN with temporal-spatial-frequency loss (TSF-MSE) to reconstruct high-sensitivity EEG signals, preserving critical neural features and improving classification performance.

Transformers have recently been adopted for 1D SR tasks due to their ability to capture long-range dependencies. \cite{iversen2025t2sr} introduced T2SR, a transformer-based model for smart meter energy data. The model achieved superior performance in terms of mean squared error (MSE) when compared to CNN and RNN baselines. \cite{gong2024super} addressed super-resolution in network telemetry time series with Zoom2Net, a transformer model capable of upsampling coarse network data by up to $100\times$, significantly improving downstream prediction accuracy.

Diffusion models have recently set new performance benchmarks in audio SR. \cite{liu2024audiosr} proposed AudioSR, a diffusion-based model that can upsample audio from 2--16\,kHz to 48\,kHz across domains including speech, music, and sound effects. \cite{im2025flashsr} later introduced FlashSR, which distills the diffusion process into a single step, achieving near real-time inference with performance comparable to full diffusion sampling.

State-space models, especially those leveraging recent advances such as the Mamba architecture \autocite{gu2023mamba}, have been incorporated into SR pipelines for both efficiency and accuracy. \cite{abreu2024aeromamba} developed AEROMamba, replacing attention mechanisms in a GAN-based audio SR model with Mamba blocks, leading to a $5\times$ reduction in GPU memory usage and improved perceptual quality. Similarly, \cite{lin2025msecg} proposed MSECG, an ECG super-resolution model using bidirectional Mamba layers and $10\times$ upsampling, surpassing prior CNN-based methods in both signal-to-noise ratio and reconstruction fidelity.

To capture multi-scale dynamics in nonstationary signals, \cite{yu2024adawavenet} introduced AdaWaveNet, a hybrid wavelet-deep network for both forecasting and super-resolution tasks. AdaWaveNet adaptively learns wavelet decompositions and achieves state-of-the-art performance across ten benchmark datasets.

\subsection{Contribution}
In this work, we address the critical challenge of limited data availability. We explore how synthetic data can be leveraged to train effective SR models in scenarios where real high-resolution signals are scarce. Thus, we aim to answer the following research questions:

\begin{itemize}
    \item \textbf{RQ1:} How effective are convolutional neural networks and other deep learning models for 1D signal super-resolution?

    \item \textbf{RQ2:} Does the inclusion of synthetic data contribute to the performance of models trained only on real data? And if so, how?

\end{itemize}

Our contribution includes a detailed empirical evaluation across synthetic and real datasets, a structured analysis of different training strategies, and an assessment of the role that synthetic data plays in enhancing performance.

\section{Methodology}
\subsection{Setup}
Let $x \in \mathbb{R}^{l}$ be a low resolution signal. We aim to recover a high resolution version of the signal $y \in \mathbb{R}^{h}$, where $h = rd$ and $r$  is the upsamplig ratio ($r = 5$ in our work).

Thus, the problem can be formulated as the construction of a model that can capture the distribution $p(y|x)$. We assume that $x$ and $y$ are related by the equation $y = f_{\theta}(x) + \epsilon$, where $f_{\theta}$ represents our model parametrized by $\theta$ and $\epsilon ~ \mathcal{N}(0, 1)$ is Gaussian noise.

In order to determine $\theta$, we use a mean squared error (MSE) training objective

\begin{equation}
    \ell(\mathcal{D})=\frac{1}{n} \sum_{i=1}^n\left\|y_i-f_\theta\left(x_i\right)\right\|_2^2
\end{equation}

where $\mathcal{D}=\left\{x_i, y_i\right\}_{i=1}^n$ is our training dataset containing the low-res/high-res signal pairs.

\subsection{Model Architecture}
Our model is a fully convolutional neural network designed for single-channel 1D signal super-resolution. The architecture follows an encoder–upsampler paradigm, where an initial stack of convolutional layers encodes the low-resolution signal into a higher-dimensional latent representation. This representation is then upsampled by a fixed factor using an interpolation-based method, followed by additional convolutional layers to refine the upsampled output.

The model accepts input signals of shape $(1, T)$, where $T$ is the number of time steps in the low-resolution sequence. The output is a signal of shape $(1, T \times r)$, where $r$ is the upsampling factor, set to $5$ in our experiments. All convolutional layers use ReLU activations, and padding is applied to preserve temporal dimensions where necessary. Table~\ref{tab:architecture} summarizes the architecture of the network.

\begin{table*}
    \centering
    \caption{Detailed architecture of the model. All convolutional layers have stride 1.}
    \label{tab:architecture}
    \begin{tabular}{|l|c|c|c|c|}
    \hline
    \textbf{Layer} & \textbf{Input Channels} & \textbf{Output Channels} & \textbf{Kernel Size} & \textbf{Padding} \\
    \hline
    Conv1D + ReLU & 1 & 64 & 9 & 4 \\
    Conv1D + ReLU & 64 & 128 & 5 & 2 \\
    Conv1D + ReLU & 128 & 256 & 5 & 2 \\
    \hline
    Upsample 5$\times$ (linear)  & - & - & - & -\\
    Conv1D + ReLU & 256 & 128 & 5 & 2 \\
    Conv1D + ReLU & 128 & 64 & 5 & 2 \\
    Conv1D & 64 & 1 & 9 & 4 \\
    \hline
    \end{tabular} 
\end{table*}

% \begin{table}[H] % Full width table (notice the starred environment)
% 	\caption{Example two column table with fixed-width columns.}
% 	\centering % Horizontally center the table
% 	\begin{tabular}{L{0.2\linewidth} L{0.2\linewidth} R{0.15\linewidth}} % Manually specify column alignments with L{}, R{} or C{} and widths as a fixed amount, usually as a proportion of \linewidth
% 		\toprule
% 		\multicolumn{2}{c}{Location} \\
% 		\cmidrule(r){1-2}
% 		East Distance & West Distance & Count \\
% 		\midrule
% 		100km & 200km & 422 \\
% 		350km & 1000km & 1833 \\
% 		600km & 1200km & 890 \\
% 		\bottomrule
% 	\end{tabular}
%     \label{tab:architecture}
% \end{table}



% \section{Super-Resolution Models}
% \input{sections/models}

\section{Experiments and Results}
\subsection{Datasets}
The synthetic dataset is derived from \href{https://zenodo.org/records/15138853}{CoSiBD}, a controlled benchmark intended to simulate non-stationary and sparse real-world signals, split between 1000 training signals and 300 validation signals. The high resolution signals originally contain 5000 points each, and the corresponding low resolution signals, 1000 ($r = 5$).

The real dataset consists of EEG time series captured in clinical environments. We use a split of 500 signals for training and 690 signals for validation. As in the synthetic dataset, the high resolution and low resolution signals contain 5000 and 1000 points, respectively.

Finally, we use the VCTK Corpus (v0.92)\cite{yamagishi2019vctk} for validation, which is an English multi-speaker dataset comprising about 44 hours of speech from 109 native speakers with diverse accents recorded at 48 kHz. We use this dataset to evaluate how capable the models are of generalization to other domains without training on them.

\begin{figure}[h]
  \centering
  \subfloat[Example of synthetic signal]{
    \includegraphics[width=0.3\textwidth]{images/synthetic_signal.pdf}
    \label{fig:synthethic_signal}
  }
  \hfill
  \subfloat[Example of EEG signal]{
    \includegraphics[width=0.3\textwidth]{images/eeg_signal.pdf}
    \label{fig:eeg_signal}
  }
  \hfill
  \subfloat[Example of VCTK signal]{
    \includegraphics[width=0.3\textwidth]{images/vctk_signal.pdf}
    \label{fig:vctk_signal}
  }
  \caption{Signal examples}
  \label{fig:signals_examples}
\end{figure}

\subsection{Methods}
We define four experimental setups:
\begin{itemize}
  \item Model $M_1$ (Real): trained on real EEG data.
  \item Model $M_2$ (Synth): trained on synthetic data.
  \item Model $M_3$ (Mixed): trained jointly on synthetic and real EEG data.
  \item Model $M_4$ (Tunned): trained on synthetic data and fine-tuned on real EEG data.
\end{itemize}
All models share the same architecture are implemented using Pytorch\cite{pytorch}.

\subsection{Metrics}
We use the Mean Absolute error (MAE) as our main evaluation metric:

\begin{equation}
    \text{MAE}=\frac{1}{n}\sum_{i=1}^n\left\|y_i-f_\theta\left(x_i\right)\right\|_1
\end{equation}

\subsection{Experiments}
The numeric results for the MAE of each model and each dataset are shown in Tables \ref{tab:mae_eeg} and \ref{tab:mae_vctk}. We also compute relative MAE changes compared to the real baseline to quantify whether the MAE is improved or worsened.

\begin{table}[H]
    \centering
    \caption{MAE on the EEG dataset (lower is better) and relative change with respect to the real baseline.}
    \label{tab:mae_eeg}
    \begin{tabular}{lrr}
        % \hline
        Model & MAE ($10^{-2})$ & MAE change\\
        \hline
        Real   & 10.771 & - \\
        Synth  & 12.109 & +12.42\% \\
        Mixed  & 9.733 & -9.64\% \\
        Tunned & 10.684 & -0.81\% \\
        \hline
    \end{tabular}
\end{table}

\begin{table}[H]
    \centering
        \caption{MAE on the VCTK dataset (lower is better) and relative change with respect to the real baseline.}
        \label{tab:mae_vctk}
        \begin{tabular}{lrr}
        % \hline
        Model & MAE ($10^{-3})$ & MAE change\\
        \hline
        Real   & 5.918 & -\\
        Synth  & 8.794 & +48.59\%\\
        Mixed  & 5.594 & -5.48\%\\
        Tunned & 4.408 & -25.51\%\\
        \hline
    \end{tabular}
\end{table}

Figure~\ref{fig:model_comparison} shows a qualitative comparison for a section of a representative EEG signal and a VCTK signal. Each subplot overlays the original signal (black) and the model output for a specific training strategy (colored traces).

\begin{figure*}[h]
  \centering
  \subfloat[EEG signal model comparison\label{fig:eeg_model_comparison}]{
    \includegraphics[width=0.8\textwidth]{images/eeg_model_comparison_1.pdf}
  }\\[0.5em]  % vertical spacing between the two images
  \subfloat[VCTK signal model comparison\label{fig:vctk_model_comparison}]{
    \includegraphics[width=0.8\textwidth]{images/vctk_model_comparison_5.pdf}
  }
  \caption{Reconstruction comparisons}
  \label{fig:model_comparison}
\end{figure*}


% \begin{figure*}[h]
%     \centering
%     \caption{Reconstruction comparisons}
    
%     \begin{subfigure}[b]{0.8\textwidth}
%         \centering
%         \includegraphics[width=\linewidth]{images/eeg_model_comparison_1.pdf}
%         \caption{EEG signal model comparison}
%         \label{fig:eeg_model_comparison}
%     \end{subfigure}
    
%     \vspace{0.5em} % Adjust vertical spacing between the two images
    
%     \begin{subfigure}[b]{0.8\textwidth}
%         \centering
%         \includegraphics[width=\linewidth]{images/vctk_model_comparison_5.pdf}
%         \caption{VCTK signal model comparison}
%         \label{fig:vctk_model_comparison}
%     \end{subfigure}
    
%     \label{fig:model_comparison}
% \end{figure*}

% \begin{figure*}
% \centering
% \caption{Reconstruction comparisons (Original vs model outputs) on a VCTK signal.}
% \includegraphics[width=0.8\textwidth]{images/vctk_model_comparison_5.pdf}
% \label{fig:waveforms}
% \end{figure*}

% \begin{figure*}
% \centering
% \caption{Reconstruction comparisons (Original vs model outputs) on a EEG signal.}
% \includegraphics[width=0.8\textwidth]{images/eeg_model_comparison_1.pdf}
% \label{fig:waveforms}
% \end{figure*}

These results indicate that using synthetic data alone yields the worst MAE on both test sets, combining synthetic with real data (Mixed) improves performance, and pretraining on synthetic and fine-tuning on real data (Tunned) improves results marginally on the EEG dataset (0.81\% MAE reduction) and significantly on the VCTK dataset (25.51\% MAE reduction).

\section{Discussion}
The MAE results (Tables \ref{tab:mae_eeg} and \ref{tab:mae_vctk}) reveal distinct behaviors across domains. On the EEG validation set, the Mixed model achieves the largest improvement, reducing MAE by 9.64\% relative to the Real baseline. The Tunned strategy produces only a marginal gain ($-0.81$\%), while the Synth-only model underperforms by 12.42\%. On the VCTK set, however, Tunned substantially outperforms all other strategies, reducing MAE by 25.51\% relative to Real, while Mixed yields a smaller 5.48\% improvement and Synth performs the worst (+48.59\%).

These findings indicate that the benefits of synthetic data depend strongly on how it is incorporated and on the evaluation domain. While joint training with real and synthetic data (Mixed) is most effective for in-domain EEG validation, pretraining on synthetic data and fine-tuning on real data (Tunned) proves highly effective for cross-domain generalization (VCTK).

The signal visualizations in Figures~\ref{fig:eeg_model_comparison} and \ref{fig:vctk_model_comparison} further illustrate these trends. For EEG (Figure~\ref{fig:eeg_model_comparison}), the Synth model produces overly smooth reconstructions, failing to capture rapid changes and oscillatory patterns. The Real and Mixed outputs both follow the ground-truth closely, with Mixed showing slightly better alignment of sharp transitions. Tunned approximates the Real output but without clear improvement, reflecting its limited MAE reduction on EEG.

For VCTK (Figure~\ref{fig:vctk_model_comparison}), the Synth model again fails to reconstruct the original signal entirely. The Real output follows the target reasonably but misses fine-grained details, while Mixed seems to mainly reduce noise. Tunned captures the original waveform most faithfully, aligning peaks and valleys more accurately, consistent with its 25.51\% MAE reduction.

The observed behavior can be explained by:
\begin{enumerate}
    \item \textbf{Domain gap.} Synthetic signals (CoSiBD) differ in spectral and amplitude statistics from EEG and speech signals. This gap explains the poor generalization of Synth-only models.
    \item \textbf{Representation bootstrap.} Pretraining on synthetic data enables the model to learn generic structures (e.g., periodicity, transient patterns). Fine-tuning on real data adapts these representations, which is particularly beneficial when generalizing across domains (Tunned on VCTK).
    \item \textbf{Data mixing.} Joint training stabilizes learning and improves in-domain performance (EEG), but does not fully align distributions for out-of-domain generalization (VCTK).
\end{enumerate}

\subsection{Answer to research questions}
\begin{itemize}
    \item \textbf{RQ1 (Effectiveness of CNNs / deep models):} The results confirm that CNN-based SR models are capable of reconstructing fine-grained structure beyond interpolation baselines, effectively reducing MAE on both EEG and speech. However, performance varies by training strategy and evaluation domain.
    
    \item \textbf{RQ2 (Contribution of synthetic data):} Synthetic data alone is insufficient and leads to degraded performance, but when combined with real data it can improve accuracy (Mixed for EEG, Tunned for VCTK). The synthetic dataset therefore serves both as an augmentation resource and as a pretraining corpus that facilitates better generalization to unseen domains.
\end{itemize}

\subsection{Limitations}
\begin{itemize}
  \item \textbf{Single metric:} MAE captures amplitude errors but does not account for perceptual or spectral fidelity. Complementary metrics such as Log-Spectral Distance (LSD), SNR, and task-based evaluations (e.g., classification accuracy) should be included.
  \item \textbf{Architecture constraints:} We restricted our study to a CNN-based model. More expressive architectures such as transformers, diffusion models, and state-space models may yield stronger results.
\end{itemize}

\section{Conclusions and Future Work}
\begin{itemize}
    \item Synthetic-only training consistently underperforms, with large errors in both EEG (+12.42\%) and VCTK (+48.59\%) relative to Real.
    \item Mixed training yields the strongest in-domain EEG performance, reducing MAE by 9.64\% relative to Real, showing that synthetic data can complement real data in the same domain.
    \item Tunned training achieves the best out-of-domain generalization, reducing MAE on VCTK by 25.51\% relative to Real, highlighting the value of pretraining on synthetic data followed by fine-tuning on real.
    \item The effectiveness of synthetic data therefore depends on the strategy: it can augment scarce real data or bootstrap cross-domain generalization.
\end{itemize}

\paragraph{Future work.}
\begin{itemize}
  \item Extend the evaluation to spectral and perceptual metrics (LSD, SNR, listening tests) and downstream performance on classification/anomaly detection tasks.
  \item Explore advanced architectures (transformers, diffusion models, state-space models) to better capture long-range dependencies and high-frequency details.
  \item Analyze the effect of synthetic dataset design (signal complexity, noise distributions, frequency content) on downstream SR performance. 
\end{itemize}

\section*{Code availability}
The corresponding code can be found \href{https://github.com/MathIAS-USFQ/time-series-srnet}{here}.

\section*{Declaration of generative AI and AI-assisted technologies in the manuscript preparation process}
During the preparation of this work the authors used ChatGPT (GPT-5, OpenAI) in order to assist with the drafting and refinement of the manuscript text, reference lookup, and to produce illustrative LaTeX/TikZ code for the visualization of the model architecture. After using this tool, the author reviewed and edited all generated content as needed and takes full responsibility for the content of the published article.

%----------------------------------------------------------------------------------------
%	 REFERENCES
%----------------------------------------------------------------------------------------

\bibliographystyle{cas-model2-names}
% Loading bibliography database
\bibliography{refs}

%----------------------------------------------------------------------------------------

\end{document}
