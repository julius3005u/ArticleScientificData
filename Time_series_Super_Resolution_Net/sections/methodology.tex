\subsection{Setup}
Let $x \in \mathbb{R}^{l}$ be a low resolution signal. We aim to recover a high resolution version of the signal $y \in \mathbb{R}^{h}$, where $h = rd$ and $r$  is the upsamplig ratio ($r = 5$ in our work).

Thus, the problem can be formulated as the construction of a model that can capture the distribution $p(y|x)$. We assume that $x$ and $y$ are related by the equation $y = f_{\theta}(x) + \epsilon$, where $f_{\theta}$ represents our model parametrized by $\theta$ and $\epsilon ~ \mathcal{N}(0, 1)$ is Gaussian noise.

In order to determine $\theta$, we use a mean squared error (MSE) training objective

\begin{equation}
    \ell(\mathcal{D})=\frac{1}{n} \sum_{i=1}^n\left\|y_i-f_\theta\left(x_i\right)\right\|_2^2
\end{equation}

where $\mathcal{D}=\left\{x_i, y_i\right\}_{i=1}^n$ is our training dataset containing the low-res/high-res signal pairs.

\subsection{Model Architecture}
Our model is a fully convolutional neural network designed for single-channel 1D signal super-resolution. The architecture follows an encoder–upsampler paradigm, where an initial stack of convolutional layers encodes the low-resolution signal into a higher-dimensional latent representation. This representation is then upsampled by a fixed factor using an interpolation-based method, followed by additional convolutional layers to refine the upsampled output.

The model accepts input signals of shape $(1, T)$, where $T$ is the number of time steps in the low-resolution sequence. The output is a signal of shape $(1, T \times r)$, where $r$ is the upsampling factor, set to $5$ in our experiments. All convolutional layers use ReLU activations, and padding is applied to preserve temporal dimensions where necessary. Table~\ref{tab:architecture} summarizes the architecture of the network.

\begin{table*}
    \centering
    \caption{Detailed architecture of the model. All convolutional layers have stride 1.}
    \label{tab:architecture}
    \begin{tabular}{|l|c|c|c|c|}
    \hline
    \textbf{Layer} & \textbf{Input Channels} & \textbf{Output Channels} & \textbf{Kernel Size} & \textbf{Padding} \\
    \hline
    Conv1D + ReLU & 1 & 64 & 9 & 4 \\
    Conv1D + ReLU & 64 & 128 & 5 & 2 \\
    Conv1D + ReLU & 128 & 256 & 5 & 2 \\
    \hline
    Upsample 5$\times$ (linear)  & - & - & - & -\\
    Conv1D + ReLU & 256 & 128 & 5 & 2 \\
    Conv1D + ReLU & 128 & 64 & 5 & 2 \\
    Conv1D & 64 & 1 & 9 & 4 \\
    \hline
    \end{tabular} 
\end{table*}

% \begin{table}[H] % Full width table (notice the starred environment)
% 	\caption{Example two column table with fixed-width columns.}
% 	\centering % Horizontally center the table
% 	\begin{tabular}{L{0.2\linewidth} L{0.2\linewidth} R{0.15\linewidth}} % Manually specify column alignments with L{}, R{} or C{} and widths as a fixed amount, usually as a proportion of \linewidth
% 		\toprule
% 		\multicolumn{2}{c}{Location} \\
% 		\cmidrule(r){1-2}
% 		East Distance & West Distance & Count \\
% 		\midrule
% 		100km & 200km & 422 \\
% 		350km & 1000km & 1833 \\
% 		600km & 1200km & 890 \\
% 		\bottomrule
% 	\end{tabular}
%     \label{tab:architecture}
% \end{table}

