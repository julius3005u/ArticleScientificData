%% LyX 2.3.7 created this file.  For more info, see http://www.lyx.org/.
%% Do not edit unless you really know what you are doing.
\documentclass[spanish,11.5pt]{article}
\usepackage[T1]{fontenc}
\usepackage[latin9,utf8x]{inputenc}
\usepackage{geometry}
\geometry{verbose,tmargin=3cm,bmargin=2.5cm,lmargin=2.5cm,rmargin=2.5cm}
\usepackage{color}
\usepackage{amstext}
\usepackage{amssymb}
\usepackage{graphicx}
\usepackage{titling}
\usepackage{fancyhdr}
\fancyhead[R]{Julio Ibarra-Fiallo}  % Encabezado a la derecha en p\'aginas pares
\fancyhead[L]{ } 
\renewcommand{\headrulewidth}{0pt} 
\pagestyle{fancy}
\usepackage{apacite}
%\bibliographystyle{apacite} 
%\makeatletter

\usepackage{caption} 
\captionsetup[table]{name=Tabla}
\captionsetup{font=small}
\usepackage{subcaption}
\date{}

%\@ifundefined{showcaptionsetup}{}{%
% \PassOptionsToPackage{caption=false}{subfig}}
%\usepackage{subfig}
%\makeatother

\usepackage{babel}
\addto\shorthandsspanish{\spanishdeactivate{~<>.}}

\setlength{\parindent}{0cm}
\setlength{\parskip}{1.5mm}

\begin{document}
\renewcommand{\abstractname}{ }

\title{\large \textbf{Reconstrucci\'on de series temporales, mediante redes neuronales: Un
enfoque de aprendizaje profundo.}}

\vspace{-8mm}
\author{\normalsize \textbf{Julio Ibarra-Fiallo, Juan A. Lara} \vspace{2mm} \\	
\emph{Programa de doctorado en computaci�n avanzada, energ��a y plasmas. } \\ \emph{Universidad de C�rdoba.} \\ \emph{z22ibfij@uco.es }}
\maketitle
\vspace{-2.2cm} 

\begin{abstract}
\normalsize
\noindent \textbf{Summary:} Reconstruction of signals represented from partial data
is a relevant problem in various fields, such as processing signals, computer vision and time series analysis.
In this work, the use of neural networks, both simple
as convolutional, to address this task. Two approaches are presented:
In the first, an autoencoder neural network is used to reconstruct
250 points of a signal from only 50 scattered points; in the second one, a convolutional neural network is used to reconstruct
5000 points of a signal from 1000 subsampled points. Results obtained show that the reconstructions based on
deep learning are as accurate as those obtained using
traditional polynomial interpolation methods, which highlights
the potential of this technique in signal reconstruction problems.

\vspace{2mm}
\noindent {\textbf{Key words}: }signal reconstruction, time series, deep learning, interpolation. 
\end{abstract}

\begin{abstract}
	\normalsize
	\noindent \textbf{Resumen:} La reconstrucci\'on de se\~nales representadas a partir de datos parciales
	es un problema relevante en diversos campos, como el procesamiento
	de se\~nales, la visi\'on por computadora y el an\'alisis de series temporales.
	En este trabajo, se explora el uso de redes neuronales, tanto simples
	como convolucionales, para abordar esta tarea. Se presentan dos enfoques:
	en el primero, se utiliza una red neuronal autoencoder para reconstruir
	250 puntos de una se\~nal a partir de solo 50 puntos dispersos; en el
	segundo, se emplea una red neuronal convolucional para reconstruir
	5000 puntos de una se\~nal a partir de 1000 puntos submuestreados. Los
	resultados obtenidos muestran que las reconstrucciones basadas en
	aprendizaje profundo son tan precisas como las obtenidas mediante
	m\'etodos tradicionales de interpolaci\'on polinomial, lo que resalta
	el potencial de esta t\'ecnica en problemas de reconstrucci\'on de se\~nales.
	
	\vspace{2mm}
	\noindent {\textbf{Palabras clave}: }reconstrucci\'on de se\~nales, series
	temporales, aprendizaje profundo, interpolaci\'on.
\end{abstract}

\section{Introducci\'on.}

La reconstrucci\'on de se\~nales a partir de datos parciales es una tarea fundamental en diversos campos, como el procesamiento de se\~nales, la visi\'on por computadora y el an\'alisis de series temporales \cite{GoodfellowBengioCourville2016, LecunbengioHinton2015}. Tradicionalmente, esta tarea se ha abordado mediante t\'ecnicas de interpolaci\'on, como los polinomios de Lagrange o los splines c\'ubicos, descomposici\'on en frecuencias \cite{MocaBarzanNagy2021}. Sin embargo, estos enfoques presentan limitaciones cuando se trata de se\~nales complejas o no lineales, o cuando los datos disponibles son escasos o ruidosos.

En  a\~nos recientes, el aprendizaje profundo ha demostrado ser una t\'ecnica poderosa para abordar una amplia gama de problemas, incluyendo el procesamiento de se\~nales y series temporales \cite{bengioDeepArch2009, MocaBarzanNagy2021}. Las redes neuronales, especialmente las redes convolucionales, han demostrado una gran capacidad para capturar patrones complejos y no lineales en los datos, lo que las convierte en una alternativa fuerte para la reconstrucci\'on de se\~nales \cite{LiuQianYang2022}.
 

\section{Metodolog\'i�a.}

\subsection{Incremento de resoluci\'on series temporales. }

Sea $S=\left\{ (t,y)/\text{ }y=s(t)\text{ }\wedge\text{ }s(t+T)=s(t)\wedge T\in\mathbb{R}\right\} $
, $S$ es una se\~nal continua, peri\'odica y $S\left|_{B}\right.$ es
la restricci\'on de $S$ cuando $t\in B\subset\mathbb{R}$ compacto
y conexo. 

\emph{Definici\'on}. Denominaremos serie temporal asociada a $S$ al
conjunto:

 $S^{*}\left|_{B}\right.=\left\{ (t_{i},y_{i})\text{ }/i=0,\dots,n\text{ }\wedge t_{i}\in\mathcal{P}(B)\right\} $
y $S^{*}\left|_{B}\right.\subset S\left|_{B}\right.$, $\mathcal{P}(B)$
es una partici\'on de $B$ uniforme.

\emph{Definici\'on. }Diremos que una serie temporal $S_{1}^{*}$ tiene
mayor resoluci\'on que una serie temporal $S_{2}^{*}$ en $B$ si $\left|S_{1}^{*}\left|_{B}\right.\right|>\left|S_{2}^{*}\left|_{B}\right.\right|$

\emph{Definici\'on.} Llamaremos un contexto de se\~nales a la categor\'i�a
$\mathcal{S}=\left\{ S/S\text{ es una se\~nal}\right\} $ y $\left|\mathcal{S}\right|<\infty$
si $\mathcal{S}\left|_{B}\right.$ todas las se\~nales de la categor\'i�a
est\'an restringidas a $B$.

El problema abordado  consiste en que dada una serie
temporal $S_{2}^{*}\left|_{B}\right.$ se busca construir otra de mayor resoluci\'on $S_{1}^{*}\left|_{B}\right.$
aprendiendo del contexto $\mathcal{S}$, estas se\~nales tienen, por ejemplo, valores aleatorios de amplitud, fase, frecuencia y per\'i�odo, 
constantes en  $B$.

En el primer enfoque se construye un modelo con arquitectura de red neuronal $\mathcal{N}$ y se
puede plantear entonces de la siguiente manera, para $v\in\mathbb{R}^{50}$,
el procesamiento $\mathcal{N}(v)\in\mathbb{R}^{250}$, entonces la
p\'erdida producida por la reconstrucci\'on $\mathcal{N}(v)$ es $\mathcal{L}=\left\Vert \mathcal{N}(v)-v^{*}\right\Vert $$^{2}$,
donde las coordenadas de $v$ son los valores de la serie temporal
$S_{2}$ de resoluci\'on $50$ y $v^{*}$ son los valores de la serie
temporal de resoluci\'on 250 asociada y $\left\Vert \cdot\right\Vert $
es la norma euclideana en $\mathbb{R}^{250}$. Para el procesamiento
por lotes de tama\~no $k$ se calcula la p\'erdida promedio como $\bar{\mathcal{L}}={\displaystyle \frac{1}{k}\sum_{i=1}^{k}\mathcal{L}_{i}}$. 

En el segundo enfoque se construye un modelo que sigue la misma l\'ogica, pero 
en la arquitectura de $\mathcal{N}$ se incluyen capas convolucionales
al procesamiento de la serie temporal de entrada y $\mathcal{N}:\mathbb{R}^{1000}\rightarrow\mathbb{R}^{5000}$.
Las arquitecturas de la red neuronal y la red convolucional se especificar\'an m\'as adelante.

\subsection{Descripci\'on del conjunto de datos .}

Para la construcci\'on de la red neuronal del primer enfoque contextual
de los modelos, se utilizaron se\~nales aleatorias sinusoidales sint\'eticas
de amplitud, fase, per\'i�odo y frecuencia constante sobre el intervalo
$B=[0,4\pi]$. En este esquema, el contexto $\mathcal{\left.\mathcal{S}\right|_{B}}$
contiene 700 series temporales de 250 puntos, este contexto proporcional
el objetivo con la resoluci\'on a alcanzar, los datos de entrada o est\'i�mulos
iniciales, son series temporales de 50 puntos, las cuales se obtuvieron
como deterioro de las series del contexto. El conjunto de validaci\'on
tiene cardinalidad 250. En el segundo enfoque, tambi\'en se utilizaron
se\~nales aleatorias sinusoidales sint\'eticas de amplitud, fase, per\'i�odo
y frecuencia constante sobre el intervalo $B=[0,4\pi]$, a diferencia
del primer enfoque las ondas transportan ondas de mucha mayor frecuencia
pero menor amplitud. El contexto $\left.\mathcal{S}\right|_{B}$ contiene
2500 series temporales de 5000 puntos, las series temporales de entrada
son de 1000 puntos deterioradas de las series del contexto. El conjunto
de validaci\'on tiene cardinalidad 750.

\subsection{Arquitectura de las redes neuronales }

%\subsubsection{Red neuronal autoencoder, primer enfoque.}

En el primer enfoque, se utiliz\'o una red neuronal autoencoder simple,
compuesta por un codificador (encoder) y un decodificador (decoder).
El codificador consta de dos capas lineales con funciones de activaci\'on
ReLU, mientras que el decodificador consta de dos capas lineales con
funciones de activaci\'on ReLU en la primera capa. La arquitectura se
muestra en la figura 1(a).

Para el entrenamiento de la red neuronal autoencoder, se utiliz\'o los
siguientes hiperpar\'ametros: \verb|n\_input = 50|,  \verb|n\_hidden = 100|, 
\verb|n\_output = 250|,  \verb|epoch = 1000|  y  \verb|learning\_rate = 0.001|.


\begin{figure}[h]
\begin{centering}
\subfloat[Red Autoencoder - Decoder ]{\centering{}\includegraphics[height=6cm]{modelo01}}\hspace{3cm}\subfloat[Red convolucional]{\begin{centering}
\includegraphics[height=6cm]{modelo02}
\par\end{centering}
}
\par\end{centering}
\caption{\label{fig:Red01}Arquitecturas primer y segundo enfoques.}

\end{figure}

%\subsubsection{Red neuronal convolucional, segundo enfoque.}

En el segundo enfoque, se utiliz\'o una red neuronal convolucional (CNN)
autoencoder. El codificador consta de dos capas convolucionales 1D
con una operaci\'on de submuestreo (max pooling) despu\'es de cada capa.
El decodificador consta de dos capas convolucionales transpuestas
1D con una operaci\'on de sobremusetreo (upsampling) despu\'es de la primera
capa. La arquitectura se muestra en la figura 1(b).

Para el entrenamiento de la red neuronal convolucional, se utiliz\'o
los siguientes hiperpar\'ametros:  
\verb|batch\_size = 128|, \verb|n\_input = 1000|,  \verb|n\_hidden = 128|, 
\verb|n\_output = 5000|,  \verb|epoch = 1000|  y  \verb|learning\_rate = 0.001|.

Ambos enfoques fueron programados con PyTorch y  usando como optimizador Adam. 
Los mejores par\'ametros fueron obtenidos por ensayo y error.

\section{Resultados.}

En la Tabla 1 se muestra el resultado de la p\'erdida de la reconstrucci\'on
basada en la red neuronal autoencoder-decoder del primer enfoque contra
la reconstrucci\'on polinomial. La reconstrucci\'on lograda con aprendizaje
se muestra muy superior a la polinomial. Se debe notar que se calcula los coeficientes de un  polinomio $p$ con todos los puntos
de  la se\~nal deteriorada, $p$ se eval\'ua en el dominio de mayor resoluci\'on. 

\begin{table}[h]
	\centering
	\begin{tabular}{lcc}
		\hline
		\textbf{Modelo} & \textbf{Entrenamiento} & \textbf{Validaci\'on} \\
		\hline
		Autoencoder & 0.0408 & 0.1167 \\
		Polinomial & 46724 & 50216 \\
		Red convolucional & 0.0156 & 0.0370 \\
		\hline
	\end{tabular}
	\caption{P\'erdida de reconstrucci\'on de los modelos.}
	\label{tab:results}
\end{table}

La Figura 2 muestras cuatro ejemplos de la reconstrucci\'on
seg\'un los esquemas propuestos de acuerdo al primer enfoque incluidos
los ajustes polinomiales. De la observaci\'on se puede conlcuir que
cuando la frecuencia es m\'as alta la red nuronal es mejor que el ajuste
polinomial, mientras que cuando no existe vibraci\'on el polinomio es
mejor.

\begin{figure}[!ht]
    \centering`  
    \begin{subfigure}[b]{0.3\textwidth}
        \includegraphics[height=2.55cm]{RedBuena01.png}
        \caption{Red neuronal, frecuencia moderada}
        \label{fig:figura1}
    \end{subfigure}
    \hfill
    \begin{subfigure}[b]{0.3\textwidth}
        \includegraphics[height=2.55cm]{RedMaoMeno02.png}
        \caption{Red enuronal, frecuencia baja }
        \label{fig:figura2}
    \end{subfigure}
    \hfill
    \begin{subfigure}[b]{0.3\textwidth}
        \includegraphics[height=2.55cm]{RedMala03.png} 
       \caption{Red neuronal, frecuencia muy baja} 
       \label{fig:figura3}
    \end{subfigure}

    \medskip
    \begin{subfigure}[b]{0.3\textwidth}
        \includegraphics[height=2.55cm]{polinomiomalo01.png}
        \caption{Polinomial, frecuencia moderada}
        \label{fig:figura4}
    \end{subfigure}
    \hfill
    \begin{subfigure}[b]{0.3\textwidth}
        \includegraphics[height=2.55cm]{polinomioMalo02.png} 
        \caption{Polinomial, frecuencia baja}
        \label{fig:figura5}     \end{subfigure}
    \hfill
    \begin{subfigure}[b]{0.3\textwidth} 
       \includegraphics[height=2.55cm]{PolinomioBueno03.png}
        \caption{Polinomial, frecuencia muy baja}
        \label{fig:figura6}
    \end{subfigure}
    \caption{Ejemplos de Reconstrucci\'on: Red y Polinomial}
\end{figure}

En la Tabla 1, tambi\'en se muestra los resultados de p\'erdida del segundo
enfoque (red neuronal convolucional) tanto en el conjunto de entrenamiento
como en el de validaci\'on. La Figura 3 muestra ejemplo de la reconstrucci\'on
de varias se\~nales del conjunto de validaci\'on.Se puede apreciar que
la reconstrucci\'on (l\'i�nea verde) sigue de cerca la se\~nal original (l\'i�nea
roja), a pesar de la gran diferencia entre el n\'umero de puntos disponibles
y el n\'umero de puntos reconstruidos, se observa tambi\'en que cuando
la onda transportada posee mayor amplitud el efecto de aliasing se
hace notorio en la reconstruci\'on.

\begin{figure}[!ht]
    \centering`  
    \begin{subfigure}[b]{0.3\textwidth}
        \includegraphics[height=2.55cm]{s500_frecuencia01.png}
        \caption{ Onda principal, frecuencia baja}
        \label{fig:figura1}
    \end{subfigure}
    \hfill
    \begin{subfigure}[b]{0.3\textwidth}
        \includegraphics[height=2.55cm]{s500_frecuencia02.png}
        \caption{Onda principal, frecuencia }
        \label{fig:figura2}
    \end{subfigure}
    \hfill
    \begin{subfigure}[b]{0.3\textwidth}
        \includegraphics[height=2.55cm]{s500_frecuencia03.png} 
       \caption{Onda transportada, frecuencia alta} 
       \label{fig:figura3}
    \end{subfigure}
	\caption{Ejemplos de reconstrucci\'on: Red convolucional}
  
\end{figure}

El trabajo sugiere que el aprendizaje profundo puede ser una alternativa
fuerte a los m\'etodos de interpolaci\'on tradicionales en situaciones
donde solo se dispone de una peque\~na fracci\'on de los datos y  la
serie temporal pertenece a un contexto conocido.

% La complejidad de las arquitecturas es baja.

Por otro lado, el enfoque basado en la red neuronal convolucional
demostr\'o ser capaz de reconstruir se\~nales a partir de datos submuestreados
con un factor de submuestreo significativo (1000 puntos a 5000 puntos).
Esto se debe a la capacidad de las redes convolucionales para capturar
patrones locales y caracter\'i�sticas relevantes en los datos, lo que
les permite generalizar de manera efectiva a regiones m\'as amplias
de la se\~nal.

Ambos enfoques presentan ventajas y limitaciones. La red neuronal
autoencoder simple es m\'as f\'acil de entrenar y requiere menos recursos
computacionales, pero puede tener dificultades para capturar patrones
complejos o no lineales en los datos. Por otro lado, la red neuronal
convolucional es m\'as poderosa y flexible, pero requiere m\'as datos
de entrenamiento y recursos computacionales. Entre las desventajas
del m\'etodo tenemos el requerimiento de informaci\'on previa para generar
un contexto $\mathcal{S}$, a partir de esto tambi\'en se puede considerar
una desventaja el hecho de que un mismo conjunto de entrada puede
generar salidas diferentes cuando $\mathcal{S}$ cambia. 

\section{Conclusiones.}

En este trabajo, se presentaron dos enfoques basados en redes neuronales
para la reconstrucci\'on de se\~nales a partir de datos parciales o submuestreados.
El primer enfoque utiliz\'o una red neuronal autoencoder simple para
reconstruir se\~nales de 250 puntos a partir de solo 50 puntos dispersos
y se lo compar\'o con la reconstrucci\'on basada en polinomios, mientras
que el segundo enfoque emple\'o una red neuronal convolucional para
reconstruir se\~nales de 5000 puntos a partir de 1000 puntos submuestreados.

Los resultados obtenidos demostraron que ambos enfoques son capaces
de reconstruir se\~nales con precisi\'on aceptable, siendo sus resultados
comparables, e incluso superiores en algunos casos, a los m\'etodos
tradicionales de interpolaci\'on polinomial. Esto resalta el potencial
del aprendizaje profundo para abordar problemas de reconstrucci\'on
de se\~nales en diversos \'ambitos, como el procesamiento de se\~nales,
la visi\'on por computadora y el an\'alisis de series temporales.

Finalmente,  se observaron ventajas y limitaciones de cada enfoque.
En general, este trabajo aprovecha la aplicaci\'on de t\'ecnicas de aprendizaje
profundo en problemas de reconstrucci\'on de series temporales.

%\begin{thebibliography}{9}
%	\bibitem{goodfellow2016deep}
%	Goodfellow, I., Bengio, Y., \& Courville, A. (2016).
%	\emph{Deep learning}. MIT press.
%	
%	\bibitem{lecun2015deep}
%	LeCun, Y., Bengio, Y., \& Hinton, G. (2015).
%	Deep learning.
%	\emph{Nature}, \emph{521}(7553), 436-444.
%	
%	\bibitem{bengio2009learning}
%	Bengio, Y. (2009).
%	Learning deep architectures for AI.
%	\emph{Foundations and trends in Machine Learning}, \emph{2}(1), 1-127.
%	
%	\bibitem{moca2021nature}
%	Moca, V.V., B\'i�rzan, H., Nagy-Dab\'i�can, A. et al. Time-frequency super-resolution with superlets.  
%	\emph{Nat Commu}, \emph{12}, 337.
%	https://doi.org/10.1038/s41467-020-20539-9	
%	
%	\bibitem{huiyurong2022electronic}
%	Liu, H.; Qian, Y.; Yang, G.; Jiang, H (2022). Super-Resolution Reconstruction Model of Spatiotemporal Fusion Remote Sensing Image Based on Double Branch Texture Transformers and Feedback Mechanism. 
%	\emph{Electronics} , \emph{11}, 2497. 
%	https://doi.org/10.3390/electronics11162497 
%\end{thebibliography}

\bibliographystyle{apacite}
\bibliography{export}

\end{document}
